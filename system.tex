\documentclass[12pt,twoside,a5paper]{report}%
\usepackage[margin=2.5cm]{geometry}
%\setlength{\columnsep}{1cm}
\usepackage{grbbridge}
\usepackage{units}
\usepackage{indentfirst}
\usepackage{url}
\usepackage{enumerate}
\newcommand{\gap}{\vspace{\baselineskip}}
\begin{document}
\title{Bidding system based on 2/1 forcing and Gazzilli convention V0.1}
\author{Che-wei Chang and Chia-sheng Chen}
\date{\today}
\maketitle
\tableofcontents


\chapter*{Abstract}
	\addcontentsline{toc}{chapter}{Abstract}
	The system is built by Che-wei Chang and Chia-sheng Chen. All the stuffs are agreed upon us mutually. Generally, the system is a 2/1 game-forcing system while we substituted \he{1}--\sp{1}--\cl{2}, and 1M--1N--\cl{2} by Gazzilli convention, which means a type of the three below,
	\begin{enumerate}[(a)]
		\item 5-3-3-2 hands of 12-14, 15-17 and 18-20 HCP.
		\item 5 Major and 4+ Clubs with 11-16 HCP.
		\item Generally all other hands of 17+ HCP (single suiter, two suiter etc.)  
	\end{enumerate}
	The system has not been built thoroughly yet, so it is still immature. Both of us look forward to build up the system.
\chapter*{Opening}
	\addcontentsline{toc}{chapter}{Opening}
	\begin{itemize}
	\renewcommand{\labelitemi}{}
		\item \cl{1}: 
		\begin{enumerate}[(a)]
			\item 12-14 HCP, BAL.
			\item 12-21 HCP, 5+\cl{}, unless 4=4=1=4.
		\end{enumerate}
		\item \di{1}: 12-21 HCP, 5+\di{}, unless 4--4--4--1 shape and \di{} has 4 cards.
		\item \he{1}: 12-21 HCP, 5+\he{}.
		\item \sp{1}: 12-21 HCP, 5+\sp{}.
		\item 1N: 11-14 HCP (not VUL), 15-17 HCP (VUL), BAL.
		\item 2N: 20-21 HCP, BAL.
		\item (\cl{1}--[ ]--2N: 18-19 HCP, BAL.)
		\item 3N: ART, 9-15 HCP, gambling, AKQ-headed 7+ minor suit, no void, no side A, at most one side K.
		\item \cl{2}: ART, 22+ HCP or 8.5+ quick tricks, any hand.
		\item \di{2}/\he{}/\sp{}: 6-11 HCP (not VUL), 8-11 HCP (VUL), 6+ suit, at least one of the top three honor.
		\item \cl{3}/\di{}/\he{}/\sp{}: PRE, 6+ suit.
		\item \cl{4}/\di{}/\he{}/\sp{}: PRE, 7+ suit.
		\item 4N: ART, RKCB ask A.
	\end{itemize}
\chapter*{Responce to \cl{1}}
	\addcontentsline{toc}{chapter}{Responce to \cl{1}}
\chapter*{Responce to \di{1}}
	\addcontentsline{toc}{chapter}{Responce to \di{1}}
\chapter*{Responce to \he{1}/\sp{}}
	\addcontentsline{toc}{chapter}{Responce to \he{1}/\sp{}}
\chapter*{Responce to 1 No-trump}
	\addcontentsline{toc}{chapter}{Responce to 1 No-trump}
\chapter*{Responce to \cl{2}}
	\addcontentsline{toc}{chapter}{Responce to \cl{2}}
\chapter*{Responce to \di{2}/\he{}/\sp{}}
	\addcontentsline{toc}{chapter}{Responce to \di{2}/\he{}/\sp{}}
\chapter*{Responce to 2 No-trump}
	\addcontentsline{toc}{chapter}{Responce to 2 No-trump}
\chapter*{Responce to Gambling 3 No-trump}
	\addcontentsline{toc}{chapter}{Responce to Gambling 3 No-trump}
\chapter*{Overcall}
	\addcontentsline{toc}{chapter}{Overcall}
\chapter*{Against No-trump}
	\addcontentsline{toc}{chapter}{Against No-trump}
	We define 15 HCP is the strongest possible hand for weak 1N opening. That is, 13-15 HCP is a weak 1N and 14-16 HCP is a strong 1N. 
	\section*{Against weak no-trump}
		\addcontentsline{toc}{section}{Against weak no-trump}
		We use Lionel Convention in this part, both direct and balance position. Lionel Convention request any direct overcall should be at least a hand worth opening.\\
		\auctionpart{1N,?}\\
		\auctionpart{1N,p,p,?}\\
		\begin{itemize}
		\renewcommand{\labelitemi}{--}
			\item DBL: \sp{}+\he{}/\di{}/\cl{} 44+ double suit.
				\begin{itemize}
				\renewcommand{\labelitemi}{--}
					\item pass: usually 10+ HCP, all subsequent doubles for penalty.
					\item \cl{2}: p/c to the doubler's second suit.
					\item \di{2}: NF, \di{} length and \he{} tolerance.
					\item \he{2}: NF, NAT.
					\item \sp{2}: S/O.
					\item 2N: spade raise, inv., no singleton.
					\item \cl{3}/\di{}/\he{}: spade raise, inv., singleton in the bidding suit.
					\item \sp{3}: PRE.
				\end{itemize}
			\item \cl{2}: \he{}+\cl{} 44+ double suit.
			\item \di{2}: \he{}+\di{} 44+ double suit.
			\item \he{2}: NAT.
			\item \sp{2}: NAT.
			\item 2N: unusual 2NT, distributional holding in both minor suits.
			\item \cl{3}: NAT.
			\item \di{3}: NAT.
		\end{itemize}
		
	\section*{Against strong no-trump}
\chapter*{Against Big \cl{}/\di{}}
	\addcontentsline{toc}{chapter}{Against Big \cl{}/\di{}}
	We use Truscott convention here, a direct overcall is a two-suiter, shows the bidding suit and the higher touching suit. (\sp{1} shows \sp{}+\cl{}). A jump overcall is similar to the preemptive bid. Double and 1N show non-touching two suits. While big \cl{}/\di{} does not show \cl{}/\di{} suit, a \cl{1}--\cl{2}, for example, does not mean Michael cuebid. Instead, it shows a \cl{}+\di{} two-suiter.\\
	For example:\\
	\auctionpart{\cl{1}!,?}\\
	\auctionpart{\cl{1}!,p,\di{1}!,?}\\
	\begin{itemize}
	\renewcommand{\labelitemi}{--}
		\item \di{1}: \di{}+\he{} 44+.
		\item \he{1}: \he{}+\sp{} 44+.
		\item \sp{1}: \sp{}+\cl{} 44+.
		\item \cl{2}: \cl{}+\di{} 44+;
		\item \di{2}/\he{}/\sp{}: PRE.
		\item DBL: a non-touching two-suit (\he{}+\cl{}/\sp{}+\di{}) which contains the suit bidded by RHO.
		\item 1N: a non-touching two-suit which does not contain the suit bidded by RHO.
	\end{itemize}
\chapter*{Leads and Signals}
\addcontentsline{toc}{chapter}{Leads and Signals}
\end{document}
